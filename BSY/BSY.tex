\chapter[Bezpečnost systémů]{B4M36BSY \\[1ex]\Large{Bezpečnostní analýza operačních systémů, bezpečný vývoj software a bezpečnost webových aplikací. Analýza útoků a škodlivého kódu. Bezpečnost mobilních zařízení.}}

Bezpečnost má tři části, říká se jim CIA:
\begin{description}
\item[Confidentiality] Služba nesmí odhalit soukromá data
\item[Integrity] Data nesmí jít změnit 
\item[Availability] Služba musí být stabilní a dostupná
\end{description}

Návrh systému - nejprve navrhneme směrnice (policies), tj pravidla, která chceme dodržovat. Směrnice určují, kdo má k čemu přístup, kdo může co dělat atd. Pokud jsou směrnice správně navržené, pak by měl být systém bezpečný. Navrhnout správnou směrnici je těžké - např. Amazon umožňoval komukoliv přidat k účtu další kreditní kartu bez jakéhokoliv ověření identity, a pak umožňoval na základě znalosti čísla jedné karty zjistit část čísla jiné karty, což je špatně.




\section{Bezpečný vývoj software}

Začínáme tím, že si ujasníme, co to je za aplikaci, co má dělat a pro koho má být (tedy funkcionalita). Pak je potřeba vědět, z jaké strany přijde útok, jak závažné ty útoky můžou být (což je hrozně těžké odhadnout). A na základě toho se pak určíme, jestli a jak se budeme bránit - někdy je prostě nejjednodušší nedělat nic, protože dopad je malý a prevence je drahá.

Aplikaci naplánujeme, nakreslíme si diagram, a pak se snažíme najít všechny možné způsoby, jak na to půjde útočit (hrozby). Ty jsou dělené do šesti tříd - STRIDE:
\begin{description}
\item[Spoofing] Manipulace, např. s identitou (říkám, že jsem někdo jiný)
\item[Tampering] Manipulace s daty nebo s kódem
\item[Repudiation] Popření toho, že se něco stalo (např. můžu tvrdit e-shopu, že mi to ještě nedorazilo, ačkoliv už mám zboží doma)
\item[Information disclosure] Únik informací (přečtu něco, co by mi nemělo být viditlené)
\item[Denial of service] Zablokování služby
\item[Escalation of privilege] Zvýšení práv (dělám víc, než by můj účet měl mít možnost dělat)
\end{description}

Hrozby taky můžeme sepsat do \textit{attack tree}, což je sekvence akcí, které může útočnik dělat, a co tím může získat. Hrozby pak nějak ohodnotím, což je ale těžké - zajímá nás, s jakou pravděpodobností útok nastane, kolik bude stát prevence, kolik bude stát oprava následků, a my žádné z těchto čísel neumíme určit.

K tomu se vytvořily metody hodnocení, třeba DREAD od Microsoftu. Tam se v každé kategorii dává skóre na škále od 0 do 10:
\begin{description}
\item[Damage] Jak velká je škoda?
\item[Reproducibility] Jak složité je ten útok zopakovat?
\item[Exploitability] Jak zkušený musí útočník být?
\item[Affected users] Na kolik uživatelů to dopadne?
\item[Discoverability] Jak složité je to pro útočníka najít?
\end{description}

\noindent Na základě hodnocení se rozhodneme, co uděláme s každou hrozbou:
\begin{itemize}
\item Neuděláme nic, protože škoda je malá a oprava by byla drahá
\item Komponentu smažeme, protože je extra děravá, a stejně to nikdo nepoužívá
\item Opravíme to
\end{itemize}

Celkově je při vývoji vhodné kontrolovat kód (motivuje to lidi, aby ho psali hezky), spouštět appku s co nejmenšími právy (když se to vysype, bude menší škoda), mít bezpečné výchozí nastavení (většina uživatelů to nechce vůbec řešit), mělo by to být pohodlné, protože jinak to lidi začnou obcházet (místo firewallu se připojí z mobilu, do hesla si dají counter protože má být složité a často měněné), důkladně testovat, nemíchat kód a data (to dělá bordej v XSS, SQLi atd).



\section{Postranní kanály}

Dva procesy (třeba ve virtuálních strojích) chtějí komunikovat, a vuyžívají datové kanály, které nejsou určeny k přenosu dat. To je problém, protože nemůžeme kontrolovat, že vše odpovídá směrnici. Jsou dva druhy:
\begin{itemize}
\item Storage: Dokážeme něco změnit, třeba bit v metadatech, a na základě toho přenášet informaci (např. dva procesy sdílejí disk a mohou nastavit, kde zastaví čtecí hlava, a tím komunikovat)
\item Timing: Dokážeme manipulovat, jak dlouho něco trvá, a tím přenášet informaci 
\end{itemize}

Postranní kanály nejde nikdy úplně odstranit. Ale přenesou jen malé množství dat, takže se dá třeba říct, že se chceme zbavit všech postranních kanálů, které mají větší kapacitu než 1bit za sekundu. Kanály mohou být se šumem nebo bez, ale i když tam je šum, tak se ho můžeme zbavit nějakým kódováním, i když tím ztratíme rychlost přenosu.


\subsection{Útoky na postranní kanály}

Útočník přistupuje ke službě a chce z ní vytáhnout nějaká data. Využívá toho, že služba se chová různě (různě se větví), podle toho, jaká data do ní dáme. Např. funkce kontroluje uživatelovo heslo, a vrátí \texttt{False} hned, jakmile narazí na špatné písmenko. Pak můžeme jednoduše hádat hesla tak, že vyzkoušíme všechna možná jednoznaková hesla, a jedno z nich bude trvat déle - to je to, ve kterém se začal porovnávat i další znak, a to je to, ve kterém máme správný  první znak. To rapidně zjednodušuje hádání hesel, pro desetiznakové heslo byla složitost $27^{10}$ a teď je $27\cdot 10$.

Kromě časových útoků jsou i útoky na spotřebu (čipové karty), na zvuk, chybové hlášky, odraz monitoru na zeď za uživatelem atd. Postranní útoky se těžko hledají - je to jeden z důvodů, proč se silně nedoporučuje implementovat si vlastní kryptografii. 

\paragraph{Steganografie} Posíláme tajná data tak, aby nikdo nevěděl, že je vůbec posíláme. Např. vytetované pod vlasy otroků, v least significant bitu v barvách obrázku, mrkání na videu (zpráva "torture" během vietnamské války).




\section{Modely řízení přístupu}

Matice přístupu je tabulka, ve které jsou zdroje (např. soubory) a kdo (procesy, uživatelé) k nim má jaký přístup.

\paragraph{Discretional Access Control (DAC)} Je metoda řízení přístupu na Unixu, soubor má vlastníka a daná práva pro čtení/zápis/spouštění pro vlastníka, skupinu a ostatní. Vznikl, když počítače začaly mít víc uživatelů, a ti si chtěli být jistí, že si omylem navzájem nesmažou data. Nevýhodou je, že když uživatel spustí program, tak ten program běží s právy toho uživatele. Druhou nevýhodou je, že ve velkých firmách to neumožňuje dostatečnou kontrolu. Další nevýhodou je, že vlastník souboru může přístupová práva ostatních měnit, to se nehodí zejména v armádě. Používá se spíš z historickou kontrolou.

\paragraph{Access Control list} Každý zdroj má majitele, a majitel může se zdrojem dělat nějaké činnosti dle seznamu. DAC se na pozadí implementuje jako tenhle seznam.

\paragraph{Tokeny} Konkurentem pro ACL jsou tokeny, kde každý zdroj dá přístup (nebo jiné právo) každému, kdo má správný token. Vlastností tokenů je, že si je uživatelé mohou předávat a někomu tak věnovat přístup.

\paragraph{Mandatory Access Control} vzniklo jako náhrada DAC pro armádu. Kontrolu nad právy nemají vlastníci souborů, ale někdo centrální. V armádě (USA) jsou různé úrovně tajnosti:

\begin{itemize}
\item \textbf{Unclassified}
\item \textbf{Classified}
\item \textbf{Secret} - pokud toto unikne, způsobí to ztrátu několika životů
\item \textbf{Top Secret} - pokud toto unikne, způsobí to ztrátu mnoha životů
\end{itemize}

Chceme zajistit, aby tajné dokumenty nemohl číst někdo, kdo na to nemá právo. Druhé omezení se týká zápisu: uživatelé, kteří mají přístup na úrovni Top Secret, budou vytvářet pouze soubory na úrovni Top Secret (tedy nikdo pod nimi si to nepřečte). To je tam proto, aby nešly vynášet přísně tajné informace.

Je to dost těžké naimplementovat, protože systém má současně otevřeno víc souborů, a pokud je jeden z nich top secret, tak by se měly všechny nastavit jako top secret. Jednodušší je to dělat přes virtuální stroje, kde každý běží na jedné bezpečnostní úrovni, a mohou komunikovat (posílat si soubory) přes hlídané kanály.

\paragraph{SE (Security Enhanced) Linux} Zdroje a spotřebitelé mají značky (labels). Spotřebitel může použít zdroj jen tehdy, pokud má správnou značku. Tím lze zajistit to, že zlý proces, ačkoliv běží s právy uživatele, nemůže napáchat žádnou škodu.

\paragraph{Role-based} je systém, který se víc hodí ve firmách. Přístup se povoluje na základě role - zaměstnanec na pozici sekretářky má přístup jinam, než zaměstnanec na pozici programátora. Je to podobné jako skupiny na Unixu, ale Unix umožňuje, aby jeden uživatel byl ve více skupinách.




\section{Operační systémy}


\paragraph{Privilege escalation}

Útočník získává přístup k obyčejnému uživateli, a z něj se pak dostane dál (horizontálně k jinému uživateli, a vertikálně na vyšší úroveň přístupu). To nastává kvůli chybám nebo špatnému vynucování směrnic.

\paragraph{Požadavky na operační systém}
\begin{itemize}
\item Musíme správně vyžadovat sběrnice
\item Procesy mimo kernel nesmí mít možnost ovlivnit kernel
\item Verifikace
\end{itemize}

\textbf{Time of check / time of use} je typ útoku, kdy se kontrola dělá jen jednou, a pak už ne, což lze zneužít. Např. v linuxu když chci zapisovat do souboru, tak se ověří, jestli smím. Já si pak soubor otevřu, a pak mi root odebere právo na zápis. Ale protože už mám soubor otevřený, můžu zapisovat dál.

\textbf{Virtualizace paměti:} každý proces si myslí, že má vlastní prostor, že adresy se kterými pracuje jsou skutečné HW adresy. A kernel mapuje procesy na skutečnou paměť. K tomu používá page tables (resp. více tabulek v sobě), a to má hw podporu, jmenuje se memory management unit.

Každý proces má vlastní page table, a když se přepíná mezi procesy, tak se musí přepnout i obsah - jinak by si procesy navzájem mohly číst data. K tomu vznikly dvě úrovně přístupu, běžná a chráněná, a jen chráněná může přepínat mezi procesy, a procesy samotné běží na běžné úrovni. Tedy, pro bezpečný systém je potřeba mít minimálně dvě úrovně v hw.

Teoreticky se vytvořily \textbf{kruhy}, kde uprostřed běží kernel, a na kraji je běžný režim. Ovladače měly běžet někde mezi tím, ale časem se posunuly do kernelu, takže se typicky mluví jen o úrovních 0 a 3.

Kdy potřebuje běžný proces přepnout do kruhu 0? Kdykoliv, když dělá nějaký syscall. Implementuje se to tak, že proces uloží do nějakých registrů číslo syscallu a parametry a zavolá software interrupt. Tím se spustí routine, ta najde příslušnou funkci, kontrola se přepne do kernelu, a kernel udělá syscall, nastaví registry a přepne zpátky do procesu.

Rychlost předání kontroly kernelu a zpět je zásadní - dřív byly linuxy oproti win pomalé, protože win běžely celé v kernel módu, a linux měl grafické rozhraní v user mode.

\textbf{Trusted computer base} je všechno, čemu musím věřit, aby byl systém bezpečný (výrobce hw, prodejce, operační systém, všechny procesy v kernelu...). Čím je menší, tím je to pro bezpečnost lepší.

Existují různé možnosti, jak psát OS a kernel:
\begin{itemize}
\item \textbf{Monolitický kernel}: Je jeden velký kernel, běží v něm všechny ovladače atd. Tenhle přístup měly dřív Windows. Bylo to rychlé, ale je těžké zaručit bezpečnost, nebo to dokonce formálně verifikovat, protože cokoliv (třeba ovladače k myši) běží v kernelu.
\item \textbf{Modulární kernel}: Současný linux i windows, kernel je docela velký, ale obsahuje jen to, co se používá
\item \textbf{Mikrokernel}: Co nejmenší kernel, jaký je vůbec možný. Musí umět: správu paměti, scheduling (přidělování času procesoru procesům), komunikaci mezi procesy. Příkladem mikrokernelu je \textit{seL4}, což je snad jediný formálně verifikovaný kernel. Mikrokernely jsou pomalejší, protože musí často předávat kontrolu mezi kruhy 0 a 3. 
\end{itemize}

Vývoj kernelu seL4 se držel správnými postupy návrhu: nejprve navrhli všechno na papíře, udělali přesnou specifikaci, a pomocí strojových nástrojů dokázali, že to je bezpečné (tj dodržuje to směrnice). Pak to napsali v haskellu, který se dobře dokazuje, a opět dokázali, že to odpovídá specifikaci. Pak přepsali funkce do C, a dokázali, že ty funkce dělají to samé, co ty v haskellu. A pak to zkompilovali do assembleru. Bezpečnost závisí na tom, že HW funguje správně, a že C kompiler funguje správně.


\begin{figure}[ht!]
\centering
\includegraphics[width=0.5\textwidth]{BSY/img/kernel-rings.png}
\end{figure}

Buffer overflow: Připravíme si nějaký prostor v paměti a necháme ho uživatelem naplnit. Přitom nehlídáme, kolik dat uživatel zapisuje. Může se tedy stát, že útočník přepsal kus paměti, do kterého už by se psát nemělo, a tím třeba přepsal ukazatel na funkci, která se měla spouštět - mohl tedy spustit tu, kterou sám chtěl. Obrana proti tomu byla, že za čtenými daty se daly speciální tokeny a pak se kontrolovalo, jestli nebyly změněny.

Heap overflow: Přijde mi to stejné, musím to prozkoumat

Return oriented programming: TODO

Structured exception handling: Funkce ošetřující výjimky jsou seřazené v linked listu. V případě chyby se postupně zkouší, jestli už jsme ve správné funkci, a ta se zavolá. Pokud je útočník schopný přepsat ukazatel na některou z funkcí a pak vyvolat chybu, může spustit vlastní kód (funkci). Obrana je taková, že se na konkrétní místo dá \textit{default handler}, a pak se zkontroluje, jestli řetězec funkcí končí v tomto handleru - pokud ne, tak někdo manipuloval s funkcemi.


\section{Izolace procesů}

Virtualizace

Sandboxing (kontejnery jako docker, a app-specific kontejnery jako v prohlížečích)

Airgap - not entirely secure, because USB sticks are still brought in

Stateless OS



\section{Bezpečnost prohlížeče}

Prohlížeče se rozsáhlostí blíží linuxovému kernelu, původně nebyly moc bezpečné, ale teď je to dost kritické, protože se hodně používají a umí skoro všechno.

Plugin: Něco, co běží mimo prohlížeč, a v prohlížeči se to jen vykresluje
Extension: Rozšíření, má přístup k porhlížeči, stránkám atd

DOM: struktura doumentu (Document Object Model)	

Same origin policy: Origin je doména, protokol (http/htps) a port.

CSRF: Cross site request forgery. Přihlásíme se do banky, dostaneme cookie. Pak jdeme na web útočníka, a ten požádá prohlížeč, aby udělal request na web banky "proveď platbu". Prohlížeč zodpovědně vyplní naši cookie, pošle to bance, a banka platbu provede. Zabrání se tomu tak, že s každým formulářem pro platbu banka pošle nějaký token (ne jako cookie), a ten token se pak ověřuje. Útočník token nemá, takže se platba neprovede.

DNS rebinding: Bezpečnost Same-origin policy závisí na bezpečnosti DNS. Pokud jsme schopní podstrčit jinou IP adresu, pak můžeme získat cookie.

HTML5 sandboxing: Dáme iframe se speciálním parametrem sandbox, a pak se obsah to iframu jen zobrazí, ale nemůže nic dělat

Content security policy: Specifikuje, z jakých zdrojů smíme stahovat obrázky, javascript atd. Jde tím třeba zakázat inline javascript. CSP specifikuje server a posílá je prohlížeči jako hlavičky. Prohlížeč pak hlídá, jestli se to dodržuje.

Strict transport security: HSTS. Po prvním HTTP requestu server řekne, že umí i HTTPS. Od té doby bude (na určitou dobu) prohlížeč umožňovat komunikaci s daným serverem pouze přes HTTPS.




\section{Chyby v návrhu protokolů}

ARP spoofing: věříme první mac adrese, která nám přijde, a tím jde udělat man in the middle.

IP sender spoofing: Můžeme libovolně měnit odesílatele v IP packetu, tím jde dělat DDoS: Několik počítačů odešle ping, kde jako odesílatel je budoucí oběť, a adresáti jsou nějaké živé stroje v síti. Ty stroje odpoví, ale budou odpovídat tomu, kdo se zeptal (to poznají podle adresy odesílatele), a tím pádem ten stroj zcela zahltí vlnou odpovědí na ping. Tomu lze předcházet tak, že poskytovatelé internetu budou kontrolovat, jestli odchozí adresy v IP hlavičkách jsou z jejich sítě, a pokud ne, jde o falešný packet a měl by se zahodit.

TCP specifikuje, jak se mají měnit výchozí sekvenční čísla v čase. Některé implementace to braly doslova, a tím šlo hádat budoucí hodnoty. A v kombinaci s manipulací s IP adresami šlo ukrást spojení. Vyřešilo se to tak, že se to číslo mírně zašumí, aby nešlo hádat.

DNS: DNS nijak nehlídá, kdo odpovídá. Nějak šlo přesvědčit server, aby přeposílal dotazy na útočníka místo na správný nadřazený DNS server.

BGP: BGP slouží k výměně informací o routování mezi autonomními systémy. Dává přednost kratším cestám (přes méně AS) a specifičtějším cestám (přesnější maska cílové sítě). BGP nijak nekontroluje, jestli je informace pravdivá, a užitečné cesty rychle posílá dál - tím jenou Pákistán shodil část internetu, když se snažil zablokovat youtube.

DNSsec: Zachovává hierarchii serverů, ale snaží se o integritu. Každý server má dva klíče: zone signing key a key signing key. Odpověď ze serveru \texttt{.cvut.cz} serveru bude podepsaná zone klíčem, a bude obsahovat ten zone klíč. Abychom ověřili zone klíč, potřebuju key signing key toho serveru, a ten mám posepsaný od serveru pro \texttt{.cz}. Klíče jsou dva proto, aby mohl server vygenerovat nový, kdyby bylo potřeba - a zone key si vygeneruje a podepíše sám, bez pomoci od \texttt{.cz} serveru.



\section{HTTPS}

Certificate pinning: Zveřejníme (skrz prohlížeč), jaké konkrétní autority můžou pro náš web vydávat certifikát.



\section{Obrana lokální sítě}

Firewall: Filtruje provoz Stateful/stateless

Bastion: Zařízení uvnitř sítě nemůžou komunikovat s internetem samy o sobě, ale musí použít bastion (specifické zařízení) jako proxy.	 		







