\chapter[Pokročilé síťové technologie]{A0M32PST \\[1ex]\Large{Směřování IP paketů. Transportní protokoly. Programově definované sítě. Virtualizace funkcí}}



Tenhle okruh obsahuje divné věci, které vůbec nesouvisí s tím, co se v předmětu dělalo. Začnu stručně těmi dvěma pojmy odzadu, a pak bude následovat to, co se fakt dělalo (a co obsahuje ty první dva pojmy). Vzhledem k tomu, že tohle je jediný síťový předmět ve státnicovém okruhu, a že jsme obor bezpečnost, tak sem přidám i věci, které v předmětu nebyly, protože je tak jako tak máme znát. Takže tady máme vše co potřebujeme vědět o sítích, plus dva záhadné pojmy navíc. Nevěnuju se tady moc aplikační vrstvě, to spíš zapadá do BSY.



\section{Programově definované sítě}

Taková věc neexistuje. Na internetu se vyskytuje podle google přesně dvakrát, jednou v zadání otázek na webu FELu, a podruhé tady na tom githubu. Správný pojem je pravděpodobně \textit{Softwarově definované sítě} (SDN), a označuje to alternativní architekturu sítě, zejména z pohledu správy.

V normální síti jsou nějaká zařízení, firewally, VPNky apod. Pokud na serveru (za všemi těmi zařízeními) přidáme novou appku, musíme změnit nastavení celé sítě, což je dost opruz. Navíc každé zařízení má svůj vlastní systém, a každý může mít úplně jiné ovládání.

SDN se to snaží řešit tak, že se přidá nová služba: \textit{SDN controller}, který nastavuje jednotlivé služby. Pro správce sítě to je prostě nadstavba nad sítí jako celkem, všechna zařízení tam jsou přehledně vidět a \textit{SDN controller} se postará o komunikaci s nimi a správné nastavení. Když se pak na server přidá nová aplikace, stačí upravit nastavení sítě na jednom místě - v SDN controlleru.

SDN je druh služby, ne konkrétní protokol, existují proprietátní řešení od cisca a jiných výrobců, nebo otevřená varianta OpenSDN s algoritmem OpenFlow.



\section{Virtualizace funkcí}

Ve větších sítích, třeba v podnikové síti pro několik poboček, se některé síťové funkce (např. DNS) opakují. Na každé pobočce je DNS server, což je zbytečné. Virtualizace síťových funkcí dělá to, že vznikne jeden server, který umí všechny ty služby, a funkcionalita na pobočkách je minimální, třeba jen firewall. Je to tak levnější, protože všechny služby v síti běží na jednom zařízení, a není potřeba mít na každé pobočce pro každou službu oddělený hardware.



\section{Úvod do sítí}

Tohle stejně všichni ví, takže jen stručně - komunikace na internetu se řídí OSI modelem, který dělí aplikace na sedm vrstev: Aplikační, Prezentační, Relační, Transportní, Síťová, Spojová, Fyzická. To je trochu moc, takže v praxi se používá TCP/IP model, který některé vrstvy slučuje a má vrstvy jen čtyři: Aplikační (Aplikační + Prezentační + Relační), Transportní, Internetová (Síťová), Lokální (Spojová + Fyzická).

Bloky dat se na každé vrstvě označují jinak (packety, rámce, apod), ale to není podstatné. Hlavní je tzv. princip zapouzdření, kdy data z aplikační vrstvy pošlu níž, tam k nim něco přidám jako hlavičku, a tak pokračuju až do fyzické vrstvy, kde dojde k vlastnímu přenosu dat (a spousty zanořených hlaviček). Na druhém stroji se pak postupně otevírají vrstvy a na základě dat v hlavičce se to zpracuje. Např. switch se podívá jen na MAC adresu a podle toho data přeposílá dál, bez toho, aniž by zjišťoval, z jaké aplikace pochází a jak jsou formátovaná. Výhodou vrstveného modelu je, že protokol v některé z vrstev lze vyměnit bez vlivu na ostatní vrstvy.

Sítě se dál dělí podle rozsahu (xxx area network) na lokální LAN, přístupové AN, metropolitní MAN, a největší (wide) WAN. Přenos dat na nejnižší (fyzické) úrovni po těchto sítích může probíhat různě, a metoda se vybere podle situace. Přesun ve volném prostoru může být směrový nebo všesměrový a může používat např. rádiové vlny nebo infračervené záření. Další možnost se přenos ve vlnovodu, např různými druhy metalických kabelů nebo optickým kabelem.

Přenosy jsou dvojího druhu: se spojením a bez spojení. Při \textbf{spojovaném přenosu} se nejprve sestaví cesta mezi koncovými body, a pak se posílají holá data. Přenos dat samotný je rychlý, ale sestavení cesty nějakou dobu trvá. Spojení navazuje třeba FrameRelay (od toho se ustupuje) a MPLS   (to budu do detailu popisovat dál). Při \textbf{přenosu bez spojení} obsahuje každý packet směrovací informace (IP adresu cíle), a do sítě se posílají hned, bez čekání na navázání cesty. Nevýhodou je, že směrování packetů za běhu je složitější.

Sítě mají různé topologie (strom, hvězda, sběrnice..) a různé architektury, např. Client-Server (webová stránka), Producent-Konzument (televizní vysílání), Peer-to-Peer (BitTorrent).



\section{Fyzická (lokální) vrstva}

Fyzická vrstva zajišťuje samotný přenos dat. Specifikuje, po jakém médiu se budou data přenášet a jak formátovaná. Na fyzické vrstvě pracují huby (opakovače/rozbočovače), které jednoduše data přepošlou, a pak switche (mosty/přepínače), které jsou v OSI modelu o vrstvu výš, a které přeposílají data do správného cíle na základě MAC adresy.

\subsection{MAC adresy}
MAC (media access control) adresa je identifikátor síťového rozhraní (karty). S MAC adresami pracuje wifi, kabelový ethernet i bluetooth. Adresa má 48 bitů a zapisuje se po šesti oktetech, tedy celkem 12 hex znaků, např.: \texttt{00-3E-E1-5F-4C-D7}. Adresa má dvě části. První část identifikuje výrobce (\texttt{00-3E-E1-xx-xx-xx} je Apple), a rozsahy jsou různě velké. Třeba Audi má adresy z celého rozsahu \texttt{01-B0-00} až \texttt{01-BF-FF}, a Apple má několik oddělených prefixů. Druhá část adresy se přiděluje při výrobě, a měla by být unikátní, aby nedocházelo ke kolizím. Ne vždy se to dodržuje, ale není to takový problém, protože MAC adresa jde ručně změnit.

Specifickou adresou je \texttt{FF-FF-FF-FF-FF-FF}, tedy samé jedničky, a ta se používá k broadcastu do celé sítě, tedy takový rámec přijmou všechna zařízení. MAC adresy lze použít i k multicastu, pak se na zařízeních nastaví \textit{skupinová} MAC adresa, a packety poslané na tuto adresu dorazí všem zařízením ve skupině.

Switche mají typicky několik portů, z některého přijde zpráva a switch se musí rozhodnout, kam ji pošle. To se dělá třeba tak, že switch udržuje tabulku MAC adres a portů, ze kterých komunikují. Když pak má data přeposlat na nějakou konkrétní adresu, tak je pošle na port, ze kterého tato adresa komunikuje. Obyčejný switch rozumí MAC adresám, ale sám žádnou nemá. Pokročilejší switche s nějakým administračním rozhraním můžou mít nějakou MAC i IP adresu, ale to je kvůli tomu rozhraní, ne kvůli funkci switche.

\subsection{Ethernet}

Ethernet zajišťuje samotný přenos dat. Specifikuje, jaký kabel se má použít, jak může být dlouhý, kolik může být po cestě opakovačů atd. Vznikal v Xeroxu a byl zveřejněn jako standard IEEE 802.3 v roce 1985. Datagramům v ethernetu se neříká packety ale rámce (frames).

\paragraph{Formát rámce}

Rámec má následující formát. Minimální délka je vždy 64 bytů, maximální délka je 1522 bytů. Mezi rámci se posílá 12 bytů mezera. V rámci se posílá i tag, ve kterém jsou dodatečné informace, například ID příslušné sítě VLAN. Tag lze úplně vypustit.

\begin{table}[ht!]
\resizebox{\textwidth}{!}{
\begin{tabular}{|c|c|c|c|c|c|c|}
\hline
Hlavička & Cílová MAC adresa & Zdrojová MAC adresa & Tag (VLAN ID) & Typ/velikost & Data & CRC\\
\hline
8 bytů & 6 bytů & 6 bytů & (4 byty) & 2 byty & 46-1500 bytů & 4 byty\\
\hline
\end{tabular}
}
\end{table}

\paragraph{CRC} Pro kontrolu dat se používá \textit{kontrolní součet} CRC. Princip je ten, že data vydělíme a zjistíme zbytek po dělení. Ten pak pošleme společně s daty a příjemce ověří, že výpočet sedí.

Binární data se dají interpretovat jako polynomy, např. $1011 = x^3 + x + 1$. To se zdá jako zbytečná komplikace, ale díky tomu se dá dělení snadno spočítat pomocí XORu. Pro CRC potřebujeme data $M(x)$ (message) a generující polynom $G(x)$. Polynom $G(x)$ je to, čím data dělíme, a je vždy předem známý. Pro zjištění \textit{zbytku} vydělíme data (zde $M(x) = 1101011111 = x^9 + x^8 + x^6 + x^4 + x^3 + x^2 + x + 1$) polynomem $G(x) = 1011 = x^3 + x + 1$. Nejprve za data přidáme tolik nul, jaký je stupeň polynomu, s tím pak pracujeme (děláme XOR v každém kroku, dokud nedostaneme zbytek 11.

\begin{lstlisting}
1101011111000
1011
0110011111000
 1011
 011111111000
  1011
  01001111000
   1011
   0010111000
     1011
     00001000
         1011
           11
\end{lstlisting}

Pokud je zbytek po dělení 11, tak víme, že pokud 11 odečteme, bude výsledné číslo dělitelné beze zbytku. V prostoru se kterým pracujeme (finite fields) se to zase dělá XORem. Data, která pošleme do sítě (transmit) tedy budou prodloužená o nuly a budou obsahovat zbytek, který jsme spočítali: $M(x) = 1101011111000 - 11 = 1101011111011$.

Příjemce si data vezme a taky je zkusí vydělit polynomem. Pokud vyjde zbytek po dělení nula, pak je pravděpodobné, že data nebyla při přenosu poškozena. Konec zprávy se jako kontrolní \textit{součet} odstraní, a zbytek jsou čistá data, která jsme chtěli poslat.
\begin{lstlisting}
1101011111011
1011
0110011111011
 1011
 011111111011
  1011
  01001111011
   1011
   0010111011
     1011
     00001011
         1011
         0000
\end{lstlisting}

Zvláštní disciplínou je výběr správného polynomu $G(x)$. Některé jsou vhodné a používané, jiné jsou zcela nefunkční. Díky tomu, jak funguje dělitelnost, můžeme pro různé polynomy matematicky ukázat, jaké chyby odhalí (kolik chybných bitů, jak daleko od sebe atd). Kromě jednotlivých chybných bitů dokáže CRC odhalit i nárazové chyby, kde je špatně celý úsek dat. V současném ethernetu se používá třeba CRC délky 32 bitů.

\paragraph{Kolize} Je potřeba zařídit, aby nedocházelo ke kolizím, tedy aby vždy vysílalo jen jedno zařízení. Zařízení sice nezačne vysílat, pokud je kanál obsazený, ale může se stát, že ve stejnou chvíli začnou vysílat dvě zařízení najednou, a pak bude signál nečitelný. Některé protokoly (typicky Master-Slave, token passing) kolize vůbec nepřipustí, protože vždy má \textit{slovo} pouze jedno zařízení. V jiných přístupech vysílající zařízení kolize detekují, a pokud nastanou, tak se náhodnou dobu čeká a pak se vysílá znovu. Přitom se napříč protokoly liší, jak se s kolizemi pracuje. Někde skončí vysílání obě strany, jinde skončí vysílání jen zařízení s nižší prioritou a to silnější pokračuje.

\subsection{Wi-Fi}
kanály, zabezpeceni

\subsection{VLAN}
VLAN je virtuální síť, která umožňuje, aby na jedné hardwarové síti existovalo víc nezávislých sítí, které do sebe vůbec nevidí. Nastavuje se to ve switchi. Každá síť má svoje VLAN ID, které se posílá s každým rámcem jako tag~-~odesílatel ho tam může dát sám, nebo ho tam přidává switch na základě toho, z jakého portu rámec přišel. Z pohledu vyšších vrstev jsou VLANky oddělené sítě, a fungují nezávisle - třeba i spanning tree protocol se počítá pro každou VLAN zvlášť. Hodí se to třeba v datacentrech, kde chce mít každý klient nějaké své servery a síť, ale datacentrum to nechce pro každého klienta fyzicky přepojovat a budovat oddělenou síť.

Na switchích, které používají VLAN, jsou dva druhy portů: trunk port a access port. \textbf{Trunk port} je port, na kterém se \textit{míchá} provoz z více VLAN najednou, např. spojení mezi dvěma switchi. V takovém provozu jsou tagy v rámcích důležité. \textbf{Access port} je takový port, který je pevně spojený s nějakou VLAN, a provoz z jiných VLAN tam není. Může to být třeba port, do kterého je připojený nějaký stolní počítač. Rámce na access portech nepotřebují tagy.

Pokud v jednom switchi jsou dvě VLANky a počítače z různých VLAN chtějí komunikovat, pak to switch sám nezvládne. Musí každou VLAN připojit do routeru, a počítače spolu budou moct komunikovat jedině přes ten router. Router nepotřebuje vědět, že je připojen dvakrát k jednomu switchi, bude to pro něj jako spojovat počítače ze dvou oddělených switchů, a o VLAN vůbec neví.

\begin{lstlisting}
SW1>                enable
SW1#                configure terminal
SW1(config)#        vlan 11
SW1(config-vlan)#   name Accounting
SW1(config-vlan)#   exit
SW1(config)#        int fa1/0
SW1(config-if)#     switchport mode access
SW1(config-if)#     switchport access vlan 11
SW1(config-if)#     end
\end{lstlisting}

Na příkladu výše je nastavení switche s označením \texttt{SW1} tak, aby vytvořil novou VLAN s id 11, jménem \textit{Accounting} a přidal rozhraní \texttt{fa1/0} jako access port pro tuto novou VLAN.

\subsection{Spanning tree protokol}

STP se používá v sítích, kde je několik switchů, které jsou navzájem propojené. Aby mohla komunikace probíhat efektivně, tak se ze všech možných spojení vyberou pouze některé (kostra grafu), a ostatní se dočasně vypnou.

V STP se vybere (podle priority) jeden switch jako ten hlavní, kořenový. Sousední switche si označí port, kterým jsou spojeny s kořenem, a sdělí svým sousedům, jak je jejich spojení s kořenem rychlé/dlouhé atd). Pokud směrovač dostane od sousedů dvě cesty do kořene, vybere si tu kratší.

\subsection{ARP}

Adress resolution protocol (ARP) slouží k mapování IPv4 adres na MAC adresy. Nějaké zařízení typicky bude chtít poslat data někomu v internetu, u koho zná IP adresu. K použití nižší vrstvy je ale potřeba znát MAC adresu. ARP funguje tak, že se zařízení zeptá \texttt{Who has 192.168.0.4} MAC broadcastem, a majitel této IP adresy odpoví. Obě zařízení si pak uloží získané MAC i IP adresy do ARP tabulky pro další použití. Pro IPv6 se používá alternativa ARPu: Neighbor discovery protocol.

\paragraph{ARP spoofing} ARP nemá žádné bezpečnostní mechanismy, a jde toho zneužít. Útočník může dvěma obětem podstrčit svou MAC adresu rychlou odpovědí na jejich dotaz \texttt{Who has xxx} (nebo nevyžádanou zprávou), a tím stáhnout veškerou komunikaci mezi oběťmi přes sebe (man in the middle útok). Někdy se tomu taky říká ARP poisoning.



\section{Síťová vrstva}

\subsection{IPv4}
IPv4 se používá od roku 1982. Zajišťuje směrování, tedy posílání packetu vždy o jeden router blíž k cíli. Je bezspojový a negarantuje doručení - to musí zajišťovat protokoly vyšších vrstev (TCP). Adresy v IPv4 mají 32 bitů a zapisují se jako 4 dekadická čísla 0-255 oddělená tečkou. Adresy jsou rozdělené autoritám podle kontinentů, a tyto autority pak přidělují adresní rozsahy firmám.

\begin{itemize}
\item AFRINIC: Afrika
\item APNIC: Jihovýchodní Asie, Austrálie
\item ARIN: Severní Amerika
\item LACNIC: Jižní Amerika až po Mexiko
\item RIPE NCC: Evropa, Rusko, Blízký Východ
\end{itemize}

\paragraph{CIDR} Původně se adresní rozsahy dělily po osmi bitech, ale i ty nejmenší sítě pak byly moc velké, a adresy začaly docházet. Proto se začaly sítě dělit na menší (subnets) tak, že se jim určila maska, tj kolik bitů na začátku adresy označuje síť (a zbytek označuje konkrétní zařízení). Síť se definuje tak, že za nejnižší adresu sítě se napíše počet bitů masky. Běžná domácí síť je 192.168.0.0/24, a jsou v ní adresy 192.168.0.0 - 192.168.0.255. Někdy se maska sítě zapisuje trochu nepřehledně ne jako počet bitů, ale jako skutečná maska, tedy síť 192.168.0.0/24 bychom reprezentovali jako 192.168.0.0/255.255.255.0.

Nejnižší adresa v podsíti je označení (název) podsítě (a kvůli jednoznačnosti se nesmí použít pro žádné zařízení) a nejvyšší adresa je broadcast. Počet adres pro zařízení v nějaké podsíti získáme tak, že nejprve spočítáme počet bitů pro adresy jako 32-(počet bitů masky), a na to umocníme dvojku. Navíc musíme odečíst dvě adresy pro název sítě a broadcast.

Některé adresy mají speciální význam - je jich hodně a stejně nemá cenu to sem dávat. Nejznámější jsou soukromé sítě 192.168.0.0/16 a loopback 127.0.0.0/8.

\paragraph{IP hlavička} Hlavička má 13 nebo 14 polí:
\begin{itemize}
\item \textbf{Version}: Vždycky 4 pro IPv4
\item \textbf{IHL}: Délka hlavičky
\item \textbf{DSCP}: Differentiated Services Code Point, specifikuje typ dat. Teď se to používá u real-time a streamování
\item \textbf{Total length}: Délka packetu (hlavička i data dohromady)
\item \textbf{Identification}: ID packetu pro případy, že data byla rozdělena na víc packetů
\item \textbf{Flags}: Informuje, zda byla data rozdělena, případně že dělena být nesmí
\item \textbf{Fragment offset}: Offset dat, pokud jsou rozdělena na víc částí
\item \textbf{Time to live}: TTL, tedy kolik routerů po cestě smí packet poslat, než bude zahozen. Slouží k tomu, aby packety nekroužily sítí. Každý router sníží TTL o jedna, ten poslední packet zahodí.
\item \textbf{Protocol}: Identifikátor protokolu z vyšší vrstvy (TCP, UDP, ICMP, EIGRP...)
\item \textbf{Header checksum}: Kontroluje integritu hlavičky
\item \textbf{Source address}: IP adresa odesílatele
\item \textbf{Destination address}: IP adresa cíle
\item \textbf{Options}: Lze vypustit, a v praxi tam často není. Může obsahovat informace k dělení packetů, streamování atd. 
\end{itemize}

\paragraph{NAT} Network Address Translation je metoda, která částečně řeší problém s nedostatkem adres. Některé IP adresy jsou veřejné, jiné jsou definované jako soukromé (třeba rozsah 192.168.0.0/16). Ke komunikaci přes internet je potřeba veřejná adresa, protože soukromé adresy budou routery zahazovat.

NAT běží v hraničním routeru soukromé sítě. Zařízení v soukromé síti chce poslat žádost serveru v Internetu. Komunikace od zařízení jde do routeru, a router si zapamatuje, které zařízení komunikuje. Pak vyšle komunikaci do Internetu se svou veřejnou IP adresou. Server žádost zpracuje a odpoví routeru, a router komunikaci přepošle správnému zařízení. Router může současně obsluhovat více zařízení, každé bude mít oddělený datový tok, ale navenek shodnou IP adresu.

Tím, že domácí zařízení mají soukromé adresy, může mít víc zařízení shodnou adresu, a není tedy potřeba tak velký adresní rozsah. NAT sice umožní zařízením uvnitř sítě komunikovat směrem ven, ale není možné z internetu kontaktovat zařízení uvnitř sítě - jeho IP adresa totiž vede jen na router, a ten pak neví, komu dál to poslat, protože soukromá IP adresa zařízení nejde zvenku zjistit (ani použít).



\subsection{IPv6}
prednaska ipv6

\subsection{DNS}

DNS patří do aplikační vrstvy, ale uvedu ho tady, protože úzce souvisí s IP adresami.

\section{Transportní vrstva}

\subsection{udp}

\subsection{tcp}
prednaska 3, sockety, principy, klouzave okno, rychlost




\section{Směrování IP paketů}
hledani nejlepsi cesty, treti vrstva, hieararchie ISP, prickove spoje a nix, interni a externi smerovaci protokoly (rip, ospf, prednaska 2)
\subsection{OSPF}



\section{MPLS}



\section{Multicast}

















