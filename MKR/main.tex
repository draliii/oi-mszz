\documentclass[10pt,a4paper]{article}
\usepackage[utf8]{inputenc}
\usepackage[czech]{babel}
\usepackage[T1]{fontenc}
\usepackage{amsmath}
\usepackage{amsfonts}
\usepackage{amssymb}
\usepackage{delarray}
\usepackage[left=2cm,right=2cm,top=2cm,bottom=2cm]{geometry}
\title{B4M01MKR - Symetrická a asymetrická kryptografie. Základní kryptosystémy. Faktorisace čísel. Hashování.}
\date{}
\begin{document}
\maketitle
\section{Symetrická a asymetrická kryptografie}
Symetrická a asymetrická kryptografie se liší vlastnostmi a způsobem použití klíčů. Zatímco v symetrické kryptografii používají obě komunikující strany stejný klíč, u asymetrické kryptografie jsou klíče dva: soukromý klíč (často značený $d$), a veřejný klíč (často značený $e$).

Symetrická kryptografie je rychlejší a proto se používá pro většinu komunikace, ale vyžaduje bezpečnou výměnu klíče. Asymetrická kryptografie je pomalejší, a podle použití klíčů umožňuje dvě věci:
\begin{itemize}
\item Kdokoliv může zašifrovat zprávu veřejným klíčem, a zprávu si pak může přečíst jen ten, komu je určena (majitel soukromého klíče)
\item Majitel klíče pošle zprávu a k ní hash zašifrovaný soukromým klíčem. Takovou zprávu si může kdokoliv přečíst, ale zprávu nelze zfalšovat (je jasné, že autorem může být jedině majitel klíče).
\end{itemize}

Díky tomu mohla vzniknout hierarchie certifikačních autorit a certifikátů. Uživatel může ověřit (podle certifikační autority, které věří), že server na druhé straně komunikace se za nikoho nevydává, a skutečně jde třeba o webovou stránku banky. S použitím asymetrické kryptografie si uživatel se serverem vymění hesla a dál mohou používat rychlejší symetrickou kryptografii.

\section{Matematika používaná v kryptografii}
\subsection{Dělitelnost, modulo, rovnice}

Základní věta aritmetiky: Každé přirozené číslo $n$ jde zapsat jako jednoznačný součin prvočísel, kde \textbf{prvočíslo} je číslo, které je dělitelné jen jedničkou a sebou samým.

\subsubsection{Počítání modulo}

\textbf{Dělení se zbytkem}: $\forall a, b \in \mathbb{Z}, a, b > 0, \exists q, r \in \mathbb{Z},$ že $ a = qb + r$, a přitom $0 \leq r \le b$, a $q, r$ jsou jednoznačné. Když je $r$ (zbytek) nulový, řekneme, že $a$ je dělitelné $b$ nebo že $b$ dělí $a$, značíme $b \mid a$. Relace dělitelnosti na $\mathbb{Z}$ je reflexivní, tranzitivní a antisymetrická, dá se znázornit Hasseho diagramem.

\textbf{Největší společný dělitel} dvou čísel $a, b$ (značíme $gcd(a,b)$) je takové číslo $d$, kde:
\begin{itemize}
\item $d \mid a \wedge d \mid b$
\item $d$ je dělitelné všemi společnými děliteli obou čísel
\item $d \geq 0$
\end{itemize}

\textbf{Nejmenší společný násobek} čísel $a, b$ (značíme $lcm(a, b)$) je nejmenší takové $d$, které je dělitelné číslem $a$ a zároveň je dělitelné číslem $b$.

Máme nějaké $n \in \mathbb{N}$, pak čísla $a, b \in \mathbb{Z}$ jsou \textbf{kongruentní modulo} $n$, pokud $n \mid (b-a)$. Kongruence se značí $a \equiv b\mod n$, je to ekvivalence a rozdělí celá čísla na $n$ tříd, kde všechny prvky ve třídě mají stejný zbytek po dělení číslem $n$. Přirozená čísla rozdělená na třídy ekvivalence podle $n$ značíme $\mathbb{Z}_n$. Relace kongruence je zachovaná při sčítání i násobení.

\subsubsection{Eulerův algoritmus}
\textbf{Euklidův algoritmus} umožňuje v lineárním čase najít $gcd(a,b)$. Dělá to tak, že rozkládá $a$ jako $a = qb + r$, a dokud je zbytek $r$ nenulový, rekurzivně spouští $gcd(b,r)$. Výhodné je použít místo tohoto postupu výpočet přes úpravy matic, kde dostaneme navíc koeficienty pro Bezoutovu větu.

\subsubsection{Bezoutova věta}
\textbf{Bezoutova věta} říká, že největší společný dělitel jde zapsat jako vztah $a,b$, tedy: $gcd(a, b) = sa + tb$, kde $s,t$ jsou celá čísla.

\subsubsection{Diofantické rovnice}
\textbf{Diofantická rovnice} je rovnice ve tvaru $ax + by = c$, kde $a,b,c \in \mathbb{Z}$. Diofantická rovnice má řešení, právě když $gcd(a,b) \mid c$. Řešení má partikulární a homogenní část.

\textbf{Př.:} V $\mathbb{Z}_{45}$ chceme řešit rovnici $12x = 6$.

\textit{Rovnici si nejprve převedeme na diofantickou, tj dostaneme $12x + 45y = 6$. Vyrobíme si jednotkovou matici o dvou sloupcích, kde první sloupec odpovídá neznámé $x$ a druhý sloupec neznámé $y$. Příslušné hodnoty v rovnici napíšeme do pravého sloupce a matici upravíme.}
\[ \left( \begin{array}{cc|c}
1 & 0 & 12\\
0 & 1 & 45
\end{array} \right)
%
\sim
%
\left( \begin{array}{cc|c}
1 & 0 & 12\\
-4 & 1 & -3
\end{array} \right)
%
\sim
%
\left( \begin{array}{cc|c}
-15 & 4 & 0\\
-4 & 1 & -3
\end{array} \right)
\]
\textit{Úpravy skončí, když se v pravém sloupci objeví nula. Druhá hodnota je největší společný dělitel čísel 12 a 45, po přenásobení řádku $-1$ vidíme, že $gcd(12,45)=3$. Protože $3 \mid 6$, bude mít diofantická rovnice řešení.}

\textit{Máme homogenní řešení $12\times(-15) + 45\times 4 = 0$ a partikulární řešení $12\times4 + 45\times(-1) = 3$. Druhou rovnici ale musíme vynásobit dvěma, abychom dostali na pravé straně 6 (odtud požadavek, aby $3\mid6$), tím dostaneme partikulární řešení $12\times8 + 45\times(-2) = 6$. Řešení se zapisuje ve tvaru $(x,y) = (x_p,y_p) + k(x_0,y_0)$, a řešení diofantické rovnice je $(x,y) = (8, -2) + k(-15,4)$. V zadání jsme ale měli řešit pouze rovnici pro $x$, a tak výsledek je $x = 8 - 15k$.}

\subsection{Grupy a jiné struktury}

\subsection{Euler-Fermatova věta}
Nechť $(G,\cdot)$ je konečná grupa o $n$ prvcích s neutrálem 1. Pak $\forall a \in G: a^n = a\cdot a\cdot a\cdots a = 1$ v $G$.

To nám říká, že když umocníme na velikost grupy, dostaneme jednotku. Pro grupy $\mathbb{Z}_n*$, které obsahují pouze čísla nesoudělná s $n$, pomůže s výpočtem velikosti grupy \textbf{Eulerova funkce} $\varphi(n)$.
\begin{itemize}
\item $\varphi(p) = p-1$ pro prvočíslo $p$.
\item $\varphi(p^k) = p^k - p^{k-1} = p^{k-1}(p-1)$ pro $k$-tou mocninu prvočísla $p$.
\item $\varphi(n\cdot m) = \varphi(n)\cdot\varphi(m)$ pro nesoudělná čísla $m,n$.
\end{itemize}

\subsection{Algoritmus opakovaných čtverců}
\subsection{Čínská věta o zbytcích}

\section{Těžké matematické problémy}
\subsection{Faktorizace čísel}
\subsection{Problém diskrétního logaritmu}

\section{Základní kryptosystémy}
\subsection{RSA}
\subsection{Diffie-Hellman}
\subsection{ElGamal}
\subsection{Eliptické křivky}

\section{Generování prvočísel}
\subsection{Fermatův test}
\subsection{Miller Rabinův test}

\section{Hashování}
\end{document}