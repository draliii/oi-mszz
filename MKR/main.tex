\documentclass[10pt,a4paper]{article}
\usepackage[utf8]{inputenc}
\usepackage[czech]{babel}
\usepackage[T1]{fontenc}
\usepackage{amsmath}
\usepackage{amsfonts}
\usepackage{amssymb}
\usepackage[left=2cm,right=2cm,top=2cm,bottom=2cm]{geometry}
\title{B4M01MKR - Symetrická a asymetrická kryptografie. Základní kryptosystémy. Faktorisace čísel. Hashování.}
\date{}
\begin{document}
\maketitle
\section{Symetrická a asymetrická kryptografie}
Symetrická a asymetrická kryptografie se liší vlastnostmi a způsobem použití klíčů. Zatímco v symetrické kryptografii používají obě komunikující strany stejný klíč, u asymetrické kryptografie jsou klíče dva: soukromý klíč (často značený $d$), a veřejný klíč (často značený $e$).

\section{Matematika používaná v kryptografii - letem světem}

\section{Těžké matematické problémy}
\subsection{Faktorizace čísel}
\subsection{Problém diskrétního logaritmu}

\section{Základní kryptosystémy}
\subsection{RSA}
\subsection{Diffie-Hellman}
\subsection{ElGamal}
\subsection{Eliptické křivky}

\section{Hashování}
\end{document}